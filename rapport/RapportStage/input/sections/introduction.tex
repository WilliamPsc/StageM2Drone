\pagenumbering{arabic}
\setcounter{page}{1}
\section{Introduction}
    \paragraph*{}
    De nos jours, l'industrie devient de plus en plus automatisée. Il est donc nécessaire d'apporter un suivi efficace à ces automatisations et aux chaines de productions automatisées afin d'intervernir et de prévenir de potentielles pannes sur le système qui peuvent occasionner une cessation de production selon la gravité.
    	
    C'est pourquoi une solution envisageable à long terme est la prévention de pannes par drones. Leurs avantages c'est qu'ils sont très mobiles, petits et très efficace dans des milieux exigus. Ils sont facile d'accès et peuvent effectuer de très nombreuses tâches variées.
    
    Nous assistons depuis quelques années au fort développement et à l'accessibilité croissante des drones sur le marché public. Ils étaient jusque là réservé à de très petites quantités de personnes telles que les militaires ou les laboratoires de recherche. Aujourd'hui, il est de plus en plus courant d'avoir chez soi un drone personnel.
    
    \paragraph*{}
    C'est pourquoi, dans le cadre de mon Master 2 LSE (Logiciels pour les Systèmes Embarqués) et également avec ma volonté de continuer sur les métiers de la Recherche, je me suis orienté vers un stage en Laboratoire sur le domaine des drones, automatisations de tâches, prévention de risques avec détection, analyse et traitement.