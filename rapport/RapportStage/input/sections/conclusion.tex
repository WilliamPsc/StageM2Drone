\section{Conclusion du stage}
    \paragraph*{}
    Le but de ce stage était d'établir une coopération entre les dispositifs présents tels que la plateforme de production, les automates de cette plateforme, le drone et le serveur. La coopération devait mener à une surveillance active des composants en analysant les possibles pannes détectées par le système.
    
    \paragraph*{}
    La première tâche du stage a été de me familiariser avec l'environnement. J'ai ainsi pu voir la plateforme industrielle en fonctionnement partiel. Et j'ai également pu découvrir le système de communication entre le Client ModBus et la plateforme faisant office de Serveur Modbus.
    Une fois ceci fait, j'ai créé un programme \textit{Java} me permettant de récupérer les valeurs des registres des automates de la plateforme. Ces valeurs sont lues puis enregistrées dans une base de données MS SQL Server 2014 hébergée sur l'ordinateur Client. Ces données doivent servir plus tard pour le drone.
    
    Dans un second temps, j'ai pris en main le drone, et relié le drone à un ordinateur externe embarqué sur le drone lui-même. Cet ordinateur embarqué, un \rpi ~3B+, me permet d'exécuter le $SDK$ fourni par le constructeur et d'accéder aux données des capteurs du drone ainsi que d'avoir une liaison avec l'autopilote du drone afin de le contrôler de manière automatique et donc d'exécuter des déplacements entre un point de départ et un point d'arrivée.
    
    Puis enfin, dans un dernier temps, je me suis penché sur l'apprentissage par un réseau de neurones, YoloV5, de deux types d'anomalies que j'ai choisi de par leurs possibilités d'être reproduites facilement ainsi que par le fait qu'il est simple de les détecter. J'ai donc créé mon dataset en prenant seulement des images avec la caméra du \rpi ~des anomalies que je cherche à détecter. Puis, j'ai entraîné le réseau sur serveur et non pas sur le \rpi ~car cela aurait été vraiment très long à cause des petites performances de calcul par rapport à ce qui est demandé pour un apprentissage de réseau de neurones. A présent, le réseau est généré et il est possible de l'exécuter sur le \rpi ~afin qu'il photographie la plateforme et détecte une anomalie par rapport à ce qu'il a appris dans cet apprentissage. Je n'ai pas eu le temps malheureusement de finir le projet. Il me reste seulement à intégrer le réseau créé par Yolo au Raspberry Pi et tester l'ensemble du code en condition réelle d'utilisation dans la pièce en prenant une photo et en voyant en temps réel s'il détecte la présence d'un tube ou d'un bouchon sur la plateforme et à plusieurs endroits du circuit si cela est possible !
    
    \paragraph*{}
    J'ai beaucoup apprécié de stage du fait de son contenu riche et diversifié. J'ai pu découvrir le fonctionnement d'un automate ainsi que du protocole de communication ModBus. Mais également, découvrir le fonctionnement d'un drone et récupérer les données de celui-ci bien qu'il ne volait pas en intérieur (trop dangereux à cause des hélices). Enfin, la découverte du fonctionnement d'un réseau de neurones m'a beaucoup intéressé car jusque là je n'avais jamais eu la possibilité de créer moi-même un réseau et de le faire apprendre sur une machine et tout cela très facilement et en peu de temps.
    
    Ce stage dans le monde de la Recherche me conforte dans mon souhait de continuer sur cette voie et d'accéder à une Thèse de Doctorat pour les années à suivre.