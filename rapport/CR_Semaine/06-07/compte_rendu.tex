\documentclass[12pt]{beamer}
\usepackage[utf8]{inputenc} % style d'écriture
\usepackage[T1]{fontenc}      % package
\usepackage[francais]{babel}  % package pour langue française
\usepackage{graphicx}
\usepackage{subcaption}
\usepackage{url}
\usepackage{color}
\usepackage{geometry}
\usepackage{amssymb}

% William PENSEC, étudiant en Master 2 LSE 2020/2021

\usetheme[secheader]{Madrid}
\beamertemplatenavigationsymbolsempty
\setbeamertemplate{frametitle continuation}{}

\title[Compte rendu de stage n\textsuperscript{o}8]{Coopération de drones dans un système hétérogène}
\subtitle{Compte rendu de stage n\textsuperscript{o}8}
\author{William \textsc{Pensec}}
%\author{William \textsc{Pensec}}
%\author{William \textsc{Pensec}}
\institute[Lab-STICC]{Lab-Sticc}
\date{07 juin 2021}

%\AtBeginSection[]
%{
%\begin{frame}<beamer>{Sommaire}
%\tableofcontents[currentsection,currentsubsection, 
%    hideothersubsections, 
%    sectionstyle=show/shaded,
%]
%\end{frame}
%}

\begin{document}
	% ---------------------------------------------------------------- %
	\begin{frame}
		\begin{titlepage}
			\begin{figure}[H]
				\centering
				\includegraphics[scale=.15]{labsticc.png}
				\hspace{3cm}
				\includegraphics[scale=.3]{ubo.png}
			\end{figure}
		\end{titlepage}
	\end{frame}
	
	% ---------------------------------------------------------------- %
	\section*{Sommaire}
	\begin{frame}
		\frametitle{Sommaire}
		\begin{center}
			\tableofcontents
		\end{center}
	\end{frame}
	%
	%% ---------------------------------------------------------------- %/
	\section{Positionnement du drone}
	\begin{frame}[allowframebreaks]
    	\frametitle{Méthodes de positionnement}
    	    \begin{block}{Méthode Bancroft}
				\begin{itemize}
    	        \setbeamertemplate{itemize items}[triangle]
				    \item Précision très médiocre
				    \item Test avec 5 et 7 ancres
				    \item A tester : Positionnement des ancres de manière exacte
				\end{itemize}
			\end{block}
	\end{frame}
	%
	% ---------------------------------------------------------------- %
	\section{Représentation de la disposition des ancres de référence}
	\begin{frame}[allowframebreaks]
    	    \begin{figure}
			    \centering
			    \includegraphics[width=0.7\textwidth]{pos.png}
			\end{figure}
	\end{frame}
	%
	%% ---------------------------------------------------------------- %
	\begin{frame}[allowframebreaks]
        \begin{table}[]
            \resizebox{\textwidth}{!}{
                \begin{tabular}{| c | c | c |}
                    \hline
                    \textbf{Nombres d'ancres} & \textbf{Temps d'exécution} & \textbf{Valeur du vecteur résultat} \\[0.15cm] \hline
                    5 ancres & entre 450 et 1200 µs & 0.68 1.37 1.46 \\[0.15cm] \hline
                    7 ancres & entre 520 et 1200 µs & 0.71 0.80 0.78 \\[0.15cm]
                    \hline
                    %Decawave position & < 100 ms & 0.65 -0.35 0.15 \\[0.15cm]
                    %\hline
                    %Réalité & / & 0.70 -0.35 0.00 \\[0.15cm]
                    %\hline
                \end{tabular}
            }
        \end{table}
	\end{frame}
	%
	%% ---------------------------------------------------------------- %
\end{document}