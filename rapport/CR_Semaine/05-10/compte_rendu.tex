\documentclass[12pt]{beamer}
\usepackage[utf8]{inputenc} % style d'écriture
\usepackage[T1]{fontenc}      % package
\usepackage[francais]{babel}  % package pour langue française
\usepackage{graphicx}
\usepackage{subcaption}
\usepackage{url}
\usepackage{color}
\usepackage{geometry}
\usepackage{amssymb}

% William PENSEC, étudiant en Master 2 LSE 2020/2021

\usetheme[secheader]{Madrid}
\beamertemplatenavigationsymbolsempty
\setbeamertemplate{frametitle continuation}{}

\title[Compte rendu de stage n\textsuperscript{o}4]{Coopération de drones dans un système hétérogène}
\subtitle{Compte rendu de stage n\textsuperscript{o}4}
\author{William \textsc{Pensec}}
%\author{William \textsc{Pensec}}
%\author{William \textsc{Pensec}}
\institute[Lab-STICC]{Lab-Sticc}
\date{10 mai 2021}

%\AtBeginSection[]
%{
%\begin{frame}<beamer>{Sommaire}
%\tableofcontents[currentsection,currentsubsection, 
%    hideothersubsections, 
%    sectionstyle=show/shaded,
%]
%\end{frame}
%}

\begin{document}
	% ---------------------------------------------------------------- %
	\begin{frame}
		\begin{titlepage}
			\begin{figure}[H]
				\centering
				\includegraphics[scale=.15]{labsticc.png}
				\hspace{3cm}
				\includegraphics[scale=.3]{ubo.png}
			\end{figure}
		\end{titlepage}
	\end{frame}
	
	% ---------------------------------------------------------------- %
	\section*{Sommaire}
	\begin{frame}
		\frametitle{Sommaire}
		\begin{center}
			\tableofcontents
		\end{center}
	\end{frame}
	%
	% ---------------------------------------------------------------- %
	\section{Positionnement via UWB}	
	\begin{frame}[allowframebreaks]
    	\frametitle{Positionnement via UWB}
    	    \begin{figure}[H]
				\centering
				\includegraphics[scale=0.7]{position.png}
			\end{figure}
			\begin{figure}[H]
				\centering
				\includegraphics[scale=0.7]{positionAndroid.png}
			\end{figure}
			\begin{figure}[H]
				\centering
				\includegraphics[scale=0.5]{data.png}
			\end{figure}
	\end{frame}
	%
	%% ---------------------------------------------------------------- %
	\section{Programme gestion positionnement \& lecture SDK}	
	\begin{frame}[allowframebreaks]
    	\frametitle{Programme gestion positionnement \& lecture SDK}
    	    \begin{exampleblock}{}
        	    \begin{itemize}
        	        \setbeamertemplate{itemize items}[triangle]
        	        \item Programme en C++
        	        \item Nécessité d'avoir 2 threads en parallèle :
        	            \begin{itemize}
        	                \setbeamertemplate{itemize items}[circle]
        	                \item 1 thread pour la gestion du positionnement
        	                \item 1 thread pour la gestion de la communication avec le drone via le SDK
        	            \end{itemize}
        	        \item Communication via USB entre la carte Decawave et le Raspberry Pi
        	        \item Communication via UART entre drone et Raspberry Pi
        	    \end{itemize}
    	    \end{exampleblock}
	\end{frame}
	%
	%% ---------------------------------------------------------------- %
	\section{A faire}
	\begin{frame}
	\frametitle{A faire}
	    \begin{alertblock}{}
    	    \begin{itemize}
    	    \setbeamertemplate{itemize items}[triangle]
    	        \item Finir le programme de gestion du positionnement et de communication avec le drone
    	        \item Positionner le drone dans la pièce selon les mesures récupérées des cartes Decawave
    	        \item Envoi et réception de données du drone selon la position du drone
    	        \item Regarder la précision nécessaire à avoir pour la plate-forme (distance des capteurs)
    	        \item Essayer avec davantage de cartes Decawave afin de voir si on peut accroitre la précision de placement
    	    \end{itemize}
	    \end{alertblock}
	\end{frame}
	%
	% ---------------------------------------------------------------- %
	%\section*{Remerciements}
	%\begin{frame}
	%\frametitle{Remerciements}
		%\begin{center}
	%	Merci pour votre attention!

		%\bigbreak
		%Avez-vous des questions?
		%\end{center}
	%\end{frame}
	% ---------------------------------------------------------------- %
	
	%\section*{Bibliographie}
	%\begin{frame}[allowframebreaks]
	%\frametitle{Bibliographie}
		%\bibliographystyle{unsrt}
		%\bibliography{bibliographie}
	%\end{frame}
	% ---------------------------------------------------------------- %
\end{document}