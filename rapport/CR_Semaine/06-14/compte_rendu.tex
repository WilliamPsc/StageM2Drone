\documentclass[12pt]{beamer}
\usepackage[utf8]{inputenc} % style d'écriture
\usepackage[T1]{fontenc}      % package
\usepackage[francais]{babel}  % package pour langue française
\usepackage{graphicx}
\usepackage{subcaption}
\usepackage{url}
\usepackage{color}
\usepackage{geometry}
\usepackage{amssymb}
\usepackage{multirow, makecell}

% William PENSEC, étudiant en Master 2 LSE 2020/2021

\usetheme[secheader]{Madrid}
\beamertemplatenavigationsymbolsempty
\setbeamertemplate{frametitle continuation}{}

\title[Compte rendu de stage n\textsuperscript{o}9]{Coopération de drones dans un système hétérogène}
\subtitle{Compte rendu de stage n\textsuperscript{o}9}
\author{William \textsc{Pensec}}
%\author{William \textsc{Pensec}}
%\author{William \textsc{Pensec}}
\institute[Lab-STICC]{Lab-Sticc}
\date{14 juin 2021}

%\AtBeginSection[]
%{
%\begin{frame}<beamer>{Sommaire}
%\tableofcontents[currentsection,currentsubsection, 
%    hideothersubsections, 
%    sectionstyle=show/shaded,
%]
%\end{frame}
%}

\begin{document}
	% ---------------------------------------------------------------- %
	\begin{frame}
		\begin{titlepage}
			\begin{figure}[H]
				\centering
				\includegraphics[scale=.15]{labsticc.png}
				\hspace{3cm}
				\includegraphics[scale=.3]{ubo.png}
			\end{figure}
		\end{titlepage}
	\end{frame}
	
	% ---------------------------------------------------------------- %
	\section*{Sommaire}
	\begin{frame}
		\frametitle{Sommaire}
		\begin{center}
			\tableofcontents
		\end{center}
	\end{frame}
	%
	% ---------------------------------------------------------------- %
	\section{Méthodologie}
	\begin{frame}
	    % représenter le drone avec les distances exactes avec les ancres
	    % déplacer le drone et faire des comparatifs des valeurs
	    % montrer les tests effectués pour voir l'amélioration des résultats
	    % faire le comparatif des résultats obtenus au fur et à mesure dans des tableaux
	    % faire des schémas de tout ce qui est possible et avec les mesures précises en comparatif
	    % calcul de l'erreur de Bancroft et Decawave par rapport à la réalité
	    % taille de la pièce : 820cm * 686 * 324
    	    \begin{figure}
			    \centering
			    \includegraphics[width=0.56\textwidth]{pos.png}
			\end{figure}
	\end{frame}%
	%% ---------------------------------------------------------------- %
	\begin{frame}[allowframebreaks]
	    \frametitle{Résultats obtenus}
        \begin{table}[]
            \resizebox{\textwidth}{!}{
                \begin{tabular}{| c | c | c |}
                    \hline
                    \textbf{Méthode} & \textbf{Temps d'exécution} & \textbf{Valeur du vecteur résultat} \\[0.15cm] \hline
                    Réalité & / & 0.76 0.43 0.26 \\[0.15cm]
                    \hline
                    Bancroft 7 ancres & entre 520 µs et 1200 µs & 0.71 0.80 0.78 \\[0.15cm]
                    \hline
                    Bancroft 5 ancres & entre 450 µs et 1200 µs & 0.68 1.37 1.46 \\[0.15cm]
                    \hline
                \end{tabular}
            }
            \caption{Semaine dernière}
        \end{table}
	\end{frame}
	%
	%% ---------------------------------------------------------------- %
	\begin{frame}[allowframebreaks]
        \begin{table}[]
            \resizebox{\textwidth}{!}{
                \begin{tabular}{| c | c | c | c |}
                    \hline
                    \textbf{Méthode} & \textbf{Temps d'exécution} & \textbf{Valeur du vecteur résultat} & \textbf{RSME\footnote{Root mean square error - Racine de l'erreur quadratique moyenne }} \\
                    \hline
                    Réalité & / & 0.76 0.43 0.26 & / \\
                    \hline
                    Bancroft & 840 µs & 0.98 -0.11 0.35 & 0.34 \\
                    \hline
                    Decawave & 840 µs & 0.78 0.38 0.26 & 0.07 \\
                    \hline
                \end{tabular}
            }
        \end{table}
        
        \begin{table}[]
            \resizebox{\textwidth}{!}{
                \begin{tabular}{| c | c | c |}
                    \hline
                    \textbf{Nombres d'ancres} & \textbf{Valeurs moyennes}\footnote{Sur 1000 itérations} & \textbf{Erreurs moyennes (RSME)}\\[0.15cm] \hline
                    Réalité & / & / \\
                    \hline
                    Bancroft & \makecell{X = 0.98\\ Y = -0.11\\ Z = 0.35} & \makecell{X = 0.223\\ Y = 0.884 \\ Z = 0.418} \\
                    \hline
                    Decawave & \makecell{X = 0.77\\ Y = 0.38\\ Z = 0.26} & \makecell{X = 0.014\\ Y = 0.068\\ Z = 0.084}\\
                    \hline
                \end{tabular}
            }
            \caption*{}
        \end{table}
	\end{frame}
	%
	%% ---------------------------------------------------------------- %
	\section{Conclusion}
    \begin{frame}[allowframebreaks]
        \begin{alertblock}{Conclusion Decawave}
            \begin{itemize}
                \setbeamertemplate{itemize item}[triangle]
                \item Calculs sur 1000 itérations du programme
                \item POS,0.72,0.27,0.3,85\textbackslash r\textbackslash n
                \item Valeur très précise en X
                \item Valeur approximative en Y
                \item Valeur approximative en Z
            \end{itemize}
        \end{alertblock}
        
        \begin{alertblock}{Conclusion Bancroft}
            \begin{itemize}
                \setbeamertemplate{itemize item}[triangle]
                \item 7 ancres
                \item Calculs sur des valeurs fixes : CD24[0.00,0.00,0.00]=0.90 DA29[0.09,0.86,0.04]=0.78 CA10[1.67,0.00,0.03]=0.91 DC30[1.45,1.04,0.07]=1.02 le\_us=3936 est[0.75,0.34,0.18,78]\textbackslash r\textbackslash n
                \item Valeurs extrêmement éloignées de la réalité 
            \end{itemize}
        \end{alertblock}
        
        \begin{alertblock}{Conclusion}
            \begin{itemize}
                \setbeamertemplate{itemize item}[triangle]
                \item Valeurs en Z à améliorer en mettant davantage de hauteur sur les ancres ($\Delta$ plus important)
                \item Précision plus importante des résultats grâce à la précision des positions de références
            \end{itemize}
        \end{alertblock}
    \end{frame}
	%
	%% ---------------------------------------------------------------- %
\end{document}