\documentclass[12pt]{beamer}
\usepackage[utf8]{inputenc} % style d'écriture
\usepackage[T1]{fontenc}      % package
\usepackage[francais]{babel}  % package pour langue française
\usepackage{graphicx}
\usepackage{subcaption}
\usepackage{url}
\usepackage{color}
\usepackage{geometry}
\usepackage{amssymb}

% William PENSEC, étudiant en Master 2 LSE 2020/2021

\usetheme[secheader]{Madrid}
\beamertemplatenavigationsymbolsempty
\setbeamertemplate{frametitle continuation}{}

\title[Compte rendu de stage n\textsuperscript{o}2]{Coopération de drones dans un système hétérogène}
\subtitle{Compte rendu de stage n\textsuperscript{o}2}
\author{William \textsc{Pensec}}
%\author{William \textsc{Pensec}}
%\author{William \textsc{Pensec}}
\institute[Lab-STICC]{Lab-Sticc}
\date{\today}

\AtBeginSection[]
{
\begin{frame}<beamer>{Sommaire}
\tableofcontents[currentsection,currentsubsection, 
    hideothersubsections, 
    sectionstyle=show/shaded,
]
\end{frame}
}

\begin{document}
	% ---------------------------------------------------------------- %
	\begin{frame}
		\begin{titlepage}
			\begin{figure}[H]
				\centering
				\includegraphics[scale=.15]{labsticc.png}
				\hspace{3cm}
				\includegraphics[scale=.3]{ubo.png}
			\end{figure}
		\end{titlepage}
	\end{frame}
	
	% ---------------------------------------------------------------- %
	\section*{Sommaire}
	\begin{frame}
		\frametitle{Sommaire}
		\begin{center}
			\tableofcontents
		\end{center}
	\end{frame}
	%
	% ---------------------------------------------------------------- %
	\section{Étude du matériel nécessaire}	
	\begin{frame}
	\frametitle{\bsc{Étude du matériel nécessaire}}
	    \begin{block}{}
			\begin{itemize}
    	    \setbeamertemplate{itemize items}[square]
				\item Raspberry Pi
				\item Batterie $\rightarrow$ voir les besoins nécessaire du système
				\item Alimentation $\rightarrow$ classique Raspberry Pi (5V - 3A)
				\item Caméra $\rightarrow$ High Quality Camera (12MPx)
				\item Lidar (M. AUTRET)
				\item Cartes de positionnement (M. AUTRET)
			\end{itemize}
		\end{block}
	\end{frame}	
	%
	% ---------------------------------------------------------------- %
	\section{Connexion Rasberry Pi 3 - Drone}	
	\begin{frame}[allowframebreaks]
    	%\frametitle{Connexion Rasberry Pi 3 - Drone}
    	    \begin{columns}
                \column{0.5\textwidth}
                    \begin{figure}
                        \includegraphics[scale=0.42]{rp3.png}
                    \end{figure}
                \column{0.5\textwidth}
                    \begin{figure}
                        \includegraphics[scale=0.4]{uart.png}
                     \end{figure}
            \end{columns}
            
            \begin{table}
                \begin{tabular}{| l | c |}
                \hline
                M100 & Raspberry Pi 3 \\
                \hline
                RXD & Tx (Pin 8) \\ 
                TXD & Rx (Pin 10) \\
                GND & GND (Pin 6) \\
                 \hline
                \end{tabular}
            \end{table}
	\end{frame}
	%
	%% ---------------------------------------------------------------- %
	\section{A faire}
	\begin{frame}
	\frametitle{A faire}
	    \begin{alertblock}{}
    	    \begin{itemize}
    	    \setbeamertemplate{itemize items}[triangle]
    	        \item Communication entre le drone et le Raspberry Pi (besoin d'une alimentation pour RPi3)
    	        \item Récupération d'informations du drone vers le Raspberry Pi
    	    \end{itemize}
	    \end{alertblock}
	\end{frame}
	%
	% ---------------------------------------------------------------- %
	\section*{Remerciements}
	\begin{frame}
	\frametitle{Remerciements}
		\begin{center}
		Merci pour votre attention!

		\bigbreak
		Avez-vous des questions?
		\end{center}
	\end{frame}
	% ---------------------------------------------------------------- %
	
	%\section*{Bibliographie}
	%\begin{frame}[allowframebreaks]
	%\frametitle{Bibliographie}
		%\bibliographystyle{unsrt}
		%\bibliography{bibliographie}
	%\end{frame}
	% ---------------------------------------------------------------- %
\end{document}